%%% Load packages
\usepackage{amssymb, gensymb, marvosym}
\usepackage{graphicx}
\usepackage[x11names]{xcolor}
\usepackage{mathrsfs}
\usepackage[shortlabels]{enumitem}
\usepackage{mathtools}
\usepackage{microtype}
\usepackage[margin=2cm]{geometry}
\usepackage{framed}
\usepackage[amsmath, thmmarks, framed]{ntheorem}
\usepackage{polski}
\usepackage{array}
\usepackage{float}
\usepackage{setspace}
\usepackage{tabto}
\usepackage{titlesec}
\usepackage{ragged2e} 
\usepackage{tasks}
\usepackage{asymptote}
\usepackage{tocloft}
\usepackage{lastpage}
\usepackage{fancyhdr} 
\usepackage{tcolorbox}
\usepackage[utf8]{inputenc} 
\usepackage{mlmodern}
\usepackage[T1]{fontenc}
\usepackage[final, colorlinks = true, linkcolor = Blue3, citecolor = Black]{hyperref} 
\usepackage{cleveref}

%%% Custom emph
\renewcommand\emph[1]{\textcolor{Blue3}{\textbf{#1}}}

%%% Lengths
\setlength\parindent{0pt}
\renewcommand{\headrulewidth}{1pt}
\renewcommand{\footrulewidth}{0pt}
\renewcommand{\baselinestretch}{1.1} 

%%% Footer and header
\pagestyle{fancy} 
\pagenumbering{arabic}
\fancyfoot[L]{} 
\fancyfoot[C]{}
\fancyfoot[R]{\thepage} 
\fancyhead[R]{}

%%% Periods after section numbers
\renewcommand{\cftsecleader}{\cftdotfill{\cftdotsep}}
\renewcommand{\cftpartaftersnum}{.}
\renewcommand{\cftsecaftersnum}{.}
\renewcommand{\cftsubsecaftersnum}{.}

%%% Cleveref fix
\crefname{equation}{}{}
\crefname{theorem}{}{}
\crefname{corollary}{}{}
\crefname{definition}{}{}

%%% Title
\titlelabel{\thetitle.\quad}

%%% Macros
\providecommand{\ol}{\overline}
\providecommand{\ul}{\underline}
\providecommand{\tl}{\widetilde}
\providecommand{\vc}{\boldsymbol}
\providecommand{\half}{\frac{1}{2}}
\providecommand{\inv}{^{-1}}
\providecommand{\NN}{\mathbb{N}}
\providecommand{\ZZ}{\mathbb{Z}}
\providecommand{\QQ}{\mathbb{Q}}
\providecommand{\RR}{\mathbb{R}}
\providecommand{\CC}{\mathbb{C}}
\providecommand{\ii}{\mathrm{i}}
\providecommand{\ee}{\mathrm{e}}
\providecommand{\dd}{\,\mathrm{d}}
\providecommand{\tsp}{\textsuperscript}
\providecommand{\tsb}{\textsubscript}
\DeclareMathOperator*{\arctg}{arctg}
\DeclareMathOperator*{\arcctg}{arcctg}

%%% Augmented matrix
\makeatletter
\renewcommand*\env@matrix[1][*\c@MaxMatrixCols c]{%
  \hskip -\arraycolsep
  \let\@ifnextchar\new@ifnextchar
  \array{#1}}
\makeatother

%%% Column type for math-mode tables
\newcolumntype{C}{>{\(}c<{\)}}

%%% Theorem-like environments

%%% Change font in parentheses
\makeatletter
\renewtheoremstyle{break}%
    {\item[\rlap{\vbox{\hbox{\hskip\labelsep \theorem@headerfont
##1\ ##2\theorem@separator}\hbox{\strut}}}]}%
    {\item[\rlap{\vbox{\hbox{\hskip\labelsep \theorem@headerfont
##1\ ##2\ \mdseries (##3)\theorem@separator}\hbox{\strut}}}]}
\makeatother

\theoremstyle{break}
\theoremheaderfont{\sffamily\bfseries\color{Blue4}}
\theorembodyfont{\mdseries}
\theoremsymbol{}
\theoremseparator{}
\newtheorem{theorem}{Twierdzenie}[section]
\tcolorboxenvironment{theorem}{colframe=Azure2, colback=Azure2!33, sharp corners = all}

% \vspace{-\belowdisplayskip}\[\] - put this hack at the end of proof if it doesn't want to behave
\theoremstyle{nonumberplain}
\theoremheaderfont{\bfseries}
\theorembodyfont{\mdseries}
\theoremsymbol{\ensuremath\square}
\theoremseparator{:}
\newtheorem{proof}{Dowód}

\theoremstyle{break}
\theoremheaderfont{\sffamily\bfseries\color{Blue4}}
\theorembodyfont{\mdseries}
\theoremsymbol{}
\theoremseparator{}
\newtheorem{corollary}{Wniosek}[theorem]
\tcolorboxenvironment{corollary}{colframe=Azure2, colback=Azure2!33, sharp corners = all}

% \theoremstyle{plain}
% \theoremheaderfont{\bfseries}
% \theorembodyfont{\mdseries}
% \theoremsymbol{}
% \theoremseparator{.}
% \newtheorem{definition}{Definicja}[section]

\theoremstyle{break}
\theoremheaderfont{\sffamily\bfseries\color{DarkOliveGreen4}}
\theorembodyfont{\mdseries}
\theoremsymbol{}
\theoremseparator{}
\newtheorem{example}[theorem]{Przykład}
\tcolorboxenvironment{example}{colframe=DarkOliveGreen2, colback=DarkOliveGreen2!33, sharp corners = all}

\theoremstyle{plain}
\theoremheaderfont{\bfseries}
\theorembodyfont{\mdseries}
\theoremsymbol{}
\theoremseparator{}
\newtheorem{problem}{Zadanie}[section]

\theoremstyle{nonumberbreak}
\theoremheaderfont{\bfseries}
\theorembodyfont{\mdseries}
\theoremsymbol{}
\theoremseparator{}
\newtheorem{solution}{}

\theoremstyle{nonumberbreak}
\theoremheaderfont{\bfseries}
\theorembodyfont{\mdseries}
\theoremsymbol{\ensuremath\square}
\theoremseparator{}
\newtheorem{solproof}{}
