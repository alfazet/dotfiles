%%% Load packages
\usepackage{amssymb, gensymb, marvosym}
\usepackage{graphicx}
\usepackage[x11names]{xcolor}
\usepackage{mathrsfs}
\usepackage[shortlabels]{enumitem}
\usepackage{mathtools}
\usepackage{microtype}
\usepackage[margin=2cm]{geometry}
\usepackage{framed}
\usepackage[amsmath, thmmarks, framed]{ntheorem}
\usepackage{polski}
\usepackage{array}
\usepackage{float}
\usepackage{caption}
\usepackage{setspace}
\usepackage{tabto}
\usepackage{titlesec}
\usepackage{ragged2e} 
\usepackage{tasks}
\usepackage{listings}
\usepackage{tocloft}
\usepackage{lastpage}
\usepackage{fancyhdr} 
\usepackage{cancel}
\usepackage[breakable]{tcolorbox}
\usepackage[utf8]{inputenc} 
\usepackage{mlmodern}
\usepackage[T1]{fontenc}
\usepackage[final, colorlinks = true, linkcolor = Blue3, citecolor = Black, urlcolor = darkgray]{hyperref} 
\usepackage{cleveref}

\usepackage{tikz}
\usetikzlibrary{arrows.meta, calc, intersections}

%% TODO: Code listings

%%% Lengths
\setlength\parindent{0pt}
\renewcommand{\headrulewidth}{1pt}
\renewcommand{\footrulewidth}{0pt}
\renewcommand{\baselinestretch}{1.1} 

%%% Footer and header
\pagestyle{fancy} 
\pagenumbering{arabic}
\fancyfoot[L]{} 
\fancyfoot[C]{\thepage}
\fancyfoot[R]{} 
\fancyhead[R]{}

%%% Periods after section numbers
\renewcommand{\cftsecleader}{\cftdotfill{\cftdotsep}}
\renewcommand{\cftpartaftersnum}{.}
\renewcommand{\cftsecaftersnum}{.}
\renewcommand{\cftsubsecaftersnum}{.}

%%% Cleveref fix
\crefname{equation}{}{}
\crefname{theorem}{}{}
\crefname{example}{}{}
\crefname{definition}{}{}

%%% Title
\titlelabel{\thetitle.\quad}

%%% Common math symbols and operators
\providecommand{\ol}{\overline}
\providecommand{\oa}{\overrightarrow}
\providecommand{\ul}{\underline}
\providecommand{\tl}{\widetilde}
\providecommand{\NN}{\mathbb{N}}
\providecommand{\ZZ}{\mathbb{Z}}
\providecommand{\QQ}{\mathbb{Q}}
\providecommand{\RR}{\mathbb{R}}
\providecommand{\CC}{\mathbb{C}}
\providecommand{\KK}{\mathbb{K}}
\providecommand{\BB}{\mathcal{B}}
\providecommand{\CC}{\mathcal{C}}
\providecommand{\LL}{\mathcal{L}}
\providecommand{\EE}{\mathcal{E}}
\providecommand{\PP}{\mathcal{P}}
\providecommand{\UU}{\mathcal{U}}
\providecommand{\VV}{\mathcal{V}}
\providecommand{\ii}{\mathrm{i}}
\providecommand{\ee}{\mathrm{e}}
\providecommand{\dd}{\,\mathrm{d}}
\providecommand{\cont}{\mathfrak{c}}

\DeclareMathOperator{\arctg}{arctg}
\DeclareMathOperator{\arcctg}{arcctg}
\DeclareMathOperator{\tr}{tr}
\DeclareMathOperator{\rank}{rank}
\DeclareMathOperator{\proj}{proj}
\DeclareMathOperator{\Hom}{Hom}
\DeclareMathOperator{\End}{End}
\DeclareMathOperator{\Ker}{Ker}
\DeclareMathOperator{\Id}{Id}
\DeclareMathOperator{\diag}{diag}
\DeclareMathOperator{\rd}{rd}
\DeclareMathOperator{\fl}{fl}
\DeclareMathOperator{\af}{af}
\DeclareMathOperator{\spann}{span}
\let\Im\relax
\DeclareMathOperator{\Im}{Im}
\let\Re\relax
\DeclareMathOperator{\Re}{Re}

\renewcommand{\vec}{\mathbf}
\renewcommand{\AA}{\mathcal{A}}

%%% Augmented matrix
\makeatletter
\renewcommand*\env@matrix[1][*\c@MaxMatrixCols c]{%
  \hskip -\arraycolsep
  \let\@ifnextchar\new@ifnextchar
  \array{#1}}
\makeatother

%%% Column type for math-mode tables
\newcolumntype{C}{>{\(}c<{\)}}

%%% Theorem-like environments
%%% Change font in parentheses to normal (not bold like the title)
\makeatletter
\renewtheoremstyle{break}%
    {\item[\rlap{\vbox{\hbox{\hskip\labelsep \theorem@headerfont
##1\ ##2\theorem@separator}\hbox{\strut}}}]}%
    {\item[\rlap{\vbox{\hbox{\hskip\labelsep \theorem@headerfont
##1\ ##2\theorem@separator \mdseries\ (##3)}\hbox{\strut}}}]}
\renewtheoremstyle{plain}%
  {\item[\hskip\labelsep \theorem@headerfont ##1\ ##2\theorem@separator]}%
  {\item[\hskip\labelsep \theorem@headerfont ##1\ ##2 \mdseries (##3)\theorem@separator]}
\renewtheoremstyle{nonumberplain}%
  {\item[\theorem@headerfont\hskip\labelsep ##1\theorem@separator]}%
  {\item[\theorem@headerfont\hskip \labelsep ##1\ \mdseries (##3)\theorem@separator]}
\makeatother

%%% Colors
% \colorlet{headcolor}{Cornsilk4!50!black}
\colorlet{framecolor}{Cornsilk3}
\colorlet{bkgcolor}{Cornsilk1!33}

% \colorlet{exheadcolor}{LavenderBlush4!50!black}
\colorlet{exframecolor}{LavenderBlush3}
\colorlet{exbkgcolor}{LavenderBlush1!33}

\colorlet{algoframecolor}{LightSteelBlue3}
\colorlet{algobkgcolor}{LightSteelBlue1!33}

\colorlet{emphcolor}{RoyalBlue4}
\colorlet{red}{IndianRed4}
%%% Custom emph
\renewcommand\emph[1]{\textcolor{emphcolor}{\textbf{#1}}}

\theoremstyle{break}
\theoremheaderfont{\bfseries}
\theorembodyfont{\mdseries}
\theoremsymbol{}
\theoremseparator{.}
\newtheorem{theorem}{Twierdzenie}[section]
\tcolorboxenvironment{theorem}{colframe=framecolor, colback=bkgcolor, sharp corners = all, breakable}

\theoremstyle{break}
\theoremheaderfont{\bfseries}
\theorembodyfont{\mdseries}
\theoremsymbol{}
\theoremseparator{}
\renewtheorem*{theorem*}{Twierdzenie}
\tcolorboxenvironment{theorem*}{colframe=framecolor, colback=bkgcolor, sharp corners = all, breakable}

\theoremstyle{break}
\theoremheaderfont{\bfseries}
\theorembodyfont{\mdseries}
\theoremsymbol{}
\theoremseparator{.}
\newtheorem{algorithm}[theorem]{Algorytm}
\tcolorboxenvironment{algorithm}{colframe=algoframecolor, colback=algobkgcolor, sharp corners = all, breakable}

% \vspace{-\belowdisplayskip}\[\] - put this hack at the end of proof if it doesn't want to behave
\theoremstyle{nonumberplain}
\theoremheaderfont{\bfseries}
\theorembodyfont{\mdseries}
\theoremsymbol{\ensuremath\square}
\theoremseparator{:}
\newtheorem{proof}{Dowód}

\theoremstyle{plain}
\theoremheaderfont{\bfseries}
\theorembodyfont{\mdseries}
\theoremsymbol{}
\theoremseparator{.}
\newtheorem{definition}[theorem]{Definicja}

\theoremstyle{break}
\theoremheaderfont{\bfseries}
\theorembodyfont{\mdseries}
\theoremsymbol{}
\theoremseparator{.}
\newtheorem{example}[theorem]{Przykład}
\tcolorboxenvironment{example}{colframe=exframecolor, colback=exbkgcolor, sharp corners = all, breakable}

\theoremstyle{plain}
\theoremheaderfont{\bfseries}
\theorembodyfont{\mdseries}
\theoremsymbol{}
\theoremseparator{.}
\newtheorem{problem}{Zadanie}

\theoremstyle{nonumberbreak}
\theoremheaderfont{\bfseries}
\theorembodyfont{\mdseries}
\theoremsymbol{}
\theoremseparator{}
\newtheorem{solution}{}

\theoremstyle{nonumberbreak}
\theoremheaderfont{\bfseries}
\theorembodyfont{\mdseries}
\theoremsymbol{\ensuremath\square}
\theoremseparator{}
\newtheorem{solproof}{}
