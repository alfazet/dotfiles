% Font
\usepackage[utf8]{inputenc} 
\usepackage{mlmodern}
\usepackage[T1]{fontenc}
% Uncomment for sans
% \renewcommand*\familydefault{\sfdefault}

% Packages
\usepackage{amssymb}
\usepackage{graphicx}
\usepackage[x11names]{xcolor}
\usepackage{mathrsfs}
\usepackage[shortlabels]{enumitem}
\usepackage{mathtools}
\usepackage{microtype}
\usepackage[margin=2.5cm]{geometry}
\usepackage{framed}
\usepackage[amsmath, thmmarks, framed]{ntheorem}
\usepackage{polski}
\usepackage{array}
\usepackage{float}
\usepackage{caption}
\usepackage{setspace}
\usepackage{tabto}
\usepackage{titlesec}
\usepackage{listings}
\usepackage{tocloft}
\usepackage{lastpage}
\usepackage{fancyhdr} 
\usepackage{cancel}
\usepackage{pdfpages}
\usepackage{transparent}
\usepackage{xifthen}
\usepackage{import}
\usepackage{printlen}
\usepackage{parskip}
\usepackage{lipsum}
\usepackage[noEnd=true,indLines=false,spaceRequire=false,rightComments=false,beginComment=//~,beginLComment=//~,endLComment=]{algpseudocodex}
\usepackage[most]{tcolorbox}
\usepackage[final, colorlinks=true, linkcolor=SlateBlue4, citecolor=black, urlcolor=darkgray]{hyperref} 
\usepackage{cleveref}

% Inkscape figures
\newcommand{\incfig}[1]{%
    \import{./figures/}{#1.pdf_tex}
}
\pdfsuppresswarningpagegroup=1

% Lengths
\setstretch{1.1}
\setlength\parindent{0pt}

% Some random code to fix a random issue
% http://tex.stackexchange.com/questions/22119/how-can-i-change-the-spacing-before-theorems-with-amsthm
\makeatletter
\def\thm@space@setup{%
  \thm@preskip=\parskip \thm@postskip=0pt
}
\makeatother

% Periods after section numbers
\renewcommand{\cftsecleader}{\cftdotfill{\cftdotsep}}
\renewcommand{\cftpartaftersnum}{.}
\renewcommand{\cftsecaftersnum}{.}
\renewcommand{\cftsubsecaftersnum}{.}

% Cleveref fix
\crefname{equation}{}{}
\crefname{theorem}{}{}
\crefname{example}{}{}
\crefname{definition}{}{}

% Title
\titlelabel{\thetitle.\quad}

% Common math symbols and operators
\providecommand{\ol}{\overline}
\providecommand{\oa}{\overrightarrow}
\providecommand{\ul}{\underline}
\providecommand{\tl}{\widetilde}

\providecommand{\NN}{\mathbb{N}}
\providecommand{\ZZ}{\mathbb{Z}}
\providecommand{\QQ}{\mathbb{Q}}
\providecommand{\RR}{\mathbb{R}}
\providecommand{\CC}{\mathbb{C}}
\providecommand{\KK}{\mathbb{K}}
\providecommand{\BB}{\mathcal{B}}
\providecommand{\CC}{\mathcal{C}}
\providecommand{\LL}{\mathcal{L}}
\providecommand{\EE}{\mathcal{E}}
\providecommand{\PP}{\mathcal{P}}
\providecommand{\UU}{\mathcal{U}}
\providecommand{\VV}{\mathcal{V}}
\providecommand{\ii}{\mathrm{i}}
\providecommand{\ee}{\mathrm{e}}
\providecommand{\dd}{\,\mathrm{d}}
\providecommand{\cont}{\mathfrak{c}}

\DeclareMathOperator{\arctg}{arctg}
\DeclareMathOperator{\arcctg}{arcctg}
\DeclareMathOperator{\tr}{tr}
\DeclareMathOperator{\rank}{rank}
\DeclareMathOperator{\proj}{proj}
\DeclareMathOperator{\Hom}{Hom}
\DeclareMathOperator{\End}{End}
\DeclareMathOperator{\Ker}{Ker}
\DeclareMathOperator{\Id}{Id}
\DeclareMathOperator{\diag}{diag}
\DeclareMathOperator{\rd}{rd}
\DeclareMathOperator{\fl}{fl}
\DeclareMathOperator{\af}{af}
\DeclareMathOperator{\spann}{span}
\let\Im\relax
\DeclareMathOperator{\Im}{Im}
\let\Re\relax
\DeclareMathOperator{\Re}{Re}

\renewcommand{\vec}[1]{\boldsymbol{\mathbf{#1}}}
\renewcommand{\AA}{\mathcal{A}}

\DeclarePairedDelimiter\abs{\lvert}{\rvert}
\DeclarePairedDelimiter\norm{\lVert}{\rVert}

\makeatletter
\renewcommand*\env@matrix[1][*\c@MaxMatrixCols c]{%
  \hskip -\arraycolsep
  \let\@ifnextchar\new@ifnextchar
  \array{#1}}
\makeatother

% Column type for math-mode tables
\newcolumntype{C}{>{\(}c<{\)}}

% enumerating (a), (b), ...
\setlist[enumerate,1]{label={(\alph*)}}

% Footer and header
\pagestyle{fancy} 
\pagenumbering{arabic}
\fancyfoot[L]{} 
\fancyfoot[C]{\thepage}
\fancyfoot[R]{} 

\fancyhead[L]{}
\fancyhead[R]{}
\renewcommand{\headrulewidth}{0pt}
\renewcommand{\footrulewidth}{0pt}

% Colors
\colorlet{thmframecolor}{Thistle3!75}
\colorlet{thmbkgcolor}{Thistle2!20}

\colorlet{defframecolor}{PeachPuff3!75}
\colorlet{defbkgcolor}{PeachPuff2!20}

\colorlet{exaframecolor}{LemonChiffon3!75}
\colorlet{exabkgcolor}{LemonChiffon1!20}

\colorlet{algframecolor}{LightSkyBlue3!75}
\colorlet{algbkgcolor}{LightSkyBlue1!20}

\colorlet{emphcolor}{RoyalBlue4}
% Custom emph
\renewcommand\emph[1]{\textcolor{emphcolor}{\textbf{#1}}}

% Theorem-like environments
% Change font in parentheses to normal (not bold like the title)
\makeatletter
\renewtheoremstyle{break}%
    {\item[\rlap{\vbox{\hbox{\hskip\labelsep \theorem@headerfont
##1\ ##2\theorem@separator}\hbox{\strut}}}]}%
    {\item[\rlap{\vbox{\hbox{\hskip\labelsep \theorem@headerfont
##1\ ##2\theorem@separator \mdseries\ (##3)}\hbox{\strut}}}]}
\renewtheoremstyle{plain}%
  {\item[\hskip\labelsep \theorem@headerfont ##1\ ##2\theorem@separator]}%
  {\item[\hskip\labelsep \theorem@headerfont ##1\ ##2 \mdseries (##3)\theorem@separator]}
\renewtheoremstyle{nonumberplain}%
  {\item[\theorem@headerfont\hskip\labelsep ##1\theorem@separator]}%
  {\item[\theorem@headerfont\hskip \labelsep ##1\ \mdseries (##3)\theorem@separator]}
\makeatother

\theoremstyle{break}
\theoremheaderfont{\bfseries}
\theorembodyfont{\mdseries}
\theoremsymbol{}
\theoremseparator{.}
\newtheorem{theorem}{Twierdzenie}[section]
\tcolorboxenvironment{theorem}{enhanced, frame hidden, borderline west={2pt}{0pt}{thmframecolor}, colframe=thmframecolor, colback=thmbkgcolor, sharp corners=all, breakable}

\theoremstyle{break}
\theoremheaderfont{\bfseries}
\theorembodyfont{\mdseries}
\theoremsymbol{}
\theoremseparator{.}
\newtheorem{lemma}[theorem]{Lemat}
\tcolorboxenvironment{lemma}{enhanced, frame hidden, borderline west={2pt}{0pt}{thmframecolor}, colframe=thmframecolor, colback=thmbkgcolor, sharp corners=all, breakable}

\theoremstyle{break}
\theoremheaderfont{\bfseries}
\theorembodyfont{\mdseries}
\theoremsymbol{}
\theoremseparator{.}
\newtheorem{definition}[theorem]{Definicja}
\tcolorboxenvironment{definition}{enhanced, frame hidden, borderline west={2pt}{0pt}{defframecolor},, colframe=defframecolor, colback=defbkgcolor, sharp corners=all, breakable}

\theoremstyle{break}
\theoremheaderfont{\bfseries}
\theorembodyfont{\mdseries}
\theoremsymbol{}
\theoremseparator{.}
\newtheorem{example}[theorem]{Przykład}
\tcolorboxenvironment{example}{enhanced, frame hidden, borderline west={2pt}{0pt}{exaframecolor}, colframe=exaframecolor, colback=exabkgcolor, sharp corners=all, breakable}

\theoremstyle{break}
\theoremheaderfont{\bfseries}
\theorembodyfont{\mdseries}
\theoremsymbol{}
\theoremseparator{.}
\newtheorem{algorithm}[theorem]{Algorytm}
\tcolorboxenvironment{algorithm}{enhanced, frame hidden, borderline west={2pt}{0pt}{algframecolor}, colback=algbkgcolor, sharp corners=all, breakable}

% \vspace{-\belowdisplayskip}\[\] - put this hack at the end of proof if the end-of-proof square isn't placed correctly 
\theoremstyle{nonumberplain}
\theoremheaderfont{\bfseries}
\theorembodyfont{\mdseries}
\theoremsymbol{\ensuremath\square}
\theoremseparator{:}
\newtheorem{proof}{Dowód}

\theoremstyle{nonumberplain}
\theoremheaderfont{\bfseries}
\theorembodyfont{\mdseries}
\theoremsymbol{}
\theoremseparator{:}
\newtheorem{solution}{Rozwiązanie}
