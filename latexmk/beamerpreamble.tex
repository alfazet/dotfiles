\usefonttheme{professionalfonts}
\usepackage{sourcesanspro}
\DeclareRobustCommand{\sbseries}{\fontseries{sb}\selectfont}
\DeclareTextFontCommand{\textsb}{\sbseries}
\usepackage{sourcecodepro}
\usepackage[T1]{fontenc}
\usepackage{MnSymbol}
\usepackage{url}
% \usepackage[final, colorlinks=true, linkcolor=SlateBlue4, citecolor=black, urlcolor=darkgray]{hyperref} 
\usepackage[cal=euler,bb=fourier]{mathalpha}
\newcommand{\hmmax}{0} % Avoid "Too many math alphabets" error
\newcommand{\bmmax}{0} % Avoid "Too many math alphabets" error
\usepackage{amssymb}
\usepackage{mathtools}
\usepackage{mathrsfs}
\usepackage{bm}
\usepackage[italic,eulergreek,nolessnomore,noparenthesis,noplusnominus,noequal,nohbar]{mathastext}
\MTsetmathskips{y}{2mu}{0mu}
\MTsetmathskips{j}{2mu}{0mu}
% \usepackage[x11names]{xcolor}

\usetheme{Boadilla}
\usecolortheme[]{seagull}
\setbeamercovered{invisible}
\setbeamertemplate{navigation symbols}{}
\setbeamertemplate{title page}[default][left,leftskip=-8pt]
\setbeamersize{text margin left=5mm,text margin right=5mm}
\usepackage[onehalfspacing]{setspace}
\usepackage{microtype}
\DisableLigatures[f]{encoding = *, family = * }
\setbeamerfont{title}{size=\huge, series=\scshape}
\setbeamerfont{author}{size=\large}
\setbeamerfont{frametitle}{series=\scshape}
\setbeamertemplate{frametitle}{\vskip3mm\MakeLowercase{\insertframetitle}}
\setbeamertemplate{frametitle continuation}{[\insertcontinuationcount]}
\setbeamerfont{itemize/enumerate subbody}{size=\normalsize} % Second level
\setbeamerfont{itemize/enumerate subsubbody}{size=\normalsize} % Third level
\setbeamerfont{block title}{series=\sbseries,size=\normalsize} % Title
\setbeamerfont{button}{size=\footnotesize}

\colorlet{BlackGray}{black!85!}
\colorlet{DarkGray}{gray!60!}
\colorlet{LightGray}{gray!30!}
\setbeamercolor{title}{fg=BlackGray}
\setbeamercolor{frametitle}{fg=BlackGray}
\setbeamercolor{normal text}{fg=BlackGray}
\setbeamercolor{itemize item}{fg=DarkGray} % Itemized list, first level
\setbeamercolor{itemize subitem}{fg=DarkGray}  % Itemized list, second level
\setbeamercolor{itemize subsubitem}{fg=DarkGray}  % Itemized list, third level
\setbeamercolor{enumerate item}{fg=DarkGray}  % Numbered list, first level
\setbeamercolor{enumerate subitem}{fg=DarkGray}  % Numbered list, second level
\setbeamercolor{enumerate subsubitem}{fg=DarkGray}  % Numbered list, third level
\setbeamercolor{block title}{fg=BlackGray,bg=LightGray} % Title
\setbeamercolor{block body}{fg=BlackGray,bg=LightGray} % Body
\setbeamercolor{footline}{fg=DarkGray}
\setbeamercolor{button}{fg=DarkGray, bg=white}

\let\oldtitle\title
\renewcommand{\title}[1]{\oldtitle[]{\MakeLowercase{#1}\vspace{-5mm}\\\color{BlackGray}{\rule{\textwidth}{2pt}}\vspace{1cm}}}

\usepackage{xparse}
\NewDocumentCommand{\information}{o g g}{%
\author[]{#2%
\IfValueT{#3}{\vspace{5mm}\\#3}%
\IfValueT{#1}{\vspace{5mm}\\\color{DarkGray}{\footnotesize Available at \url{#1}}}%
}\date[]{}}

\setbeamertemplate{itemize item}{\textbullet} % First-level item
\setbeamertemplate{itemize subitem}{\textendash} % Second-level item
\setbeamertemplate{itemize subsubitem}{\textsquare} % Third-level item
\setbeamertemplate{enumerate item}[default] % First-level item
\setbeamertemplate{enumerate subitem}{\alph{enumii}.} % Second-level item
\setbeamertemplate{enumerate subsubitem}{\roman{enumiii}.} % Third-level item

\NewDocumentCommand{\al}{o g}{%
\IfNoValueTF{#1}{\textcolor{Red1}{#2}}%
{\textcolor<#1>{Red1}{#2}}}

\NewDocumentCommand{\alg}{o g}{%
\IfNoValueTF{#1}{\textcolor{Green1}{#2}}%
{\textcolor<#1>{Green1}{#2}}}
\NewDocumentCommand{\alr}{o g}{%
\IfNoValueTF{#1}{\textcolor{Red1}{#2}}%
{\textcolor<#1>{Red1}{#2}}}
\NewDocumentCommand{\alb}{o g}{%
\IfNoValueTF{#1}{\textcolor{Blue1}{#2}}%
{\textcolor<#1>{Blue1}{#2}}}

\usepackage{booktabs}
\usepackage{multirow}
\usepackage{caption}
\captionsetup{labelformat=empty,size=normalsize}
\let\oldincludegraphics\includegraphics 
\renewcommand{\includegraphics}[2][]{\centering\oldincludegraphics[#1]{#2}}

\setbeamertemplate{blocks}[rounded][shadow=false]
\newtheorem{proposition}{Proposition}
\newtheorem{assumption}{Assumption}
\newtheorem{remark}{Remark}

\newcommand{\heading}[1]{\LARGE\scshape\color{BlackGray}\singlespacing\MakeLowercase{#1}}

\newcommand{\lastslide}{{\setbeamercolor{normal text}{bg=LightGray}\begin{frame}\end{frame}}}

\setbeamertemplate{footline}{}

% Uncomment the line below to insert slide numbers
% \setbeamertemplate{footline}[frame number]

\usepackage{polski}
\usepackage{cancel}

% Common math symbols and operators
\providecommand{\ol}{\overline}
\providecommand{\oa}{\overrightarrow}
\providecommand{\ul}{\underline}
\providecommand{\tl}{\widetilde}

\providecommand{\NN}{\mathbb{N}}
\providecommand{\ZZ}{\mathbb{Z}}
\providecommand{\QQ}{\mathbb{Q}}
\providecommand{\RR}{\mathbb{R}}
\providecommand{\CC}{\mathbb{C}}
\providecommand{\KK}{\mathbb{K}}
\providecommand{\BB}{\mathcal{B}}
\providecommand{\CC}{\mathcal{C}}
\providecommand{\LL}{\mathcal{L}}
\providecommand{\EE}{\mathcal{E}}
\providecommand{\PP}{\mathcal{P}}
\providecommand{\UU}{\mathcal{U}}
\providecommand{\VV}{\mathcal{V}}
\providecommand{\ii}{\mathrm{i}}
\providecommand{\ee}{\mathrm{e}}
\providecommand{\dd}{\,\mathrm{d}}
\providecommand{\cont}{\mathfrak{c}}

\DeclareMathOperator{\arctg}{arctg}
\DeclareMathOperator{\arcctg}{arcctg}
\DeclareMathOperator{\tr}{tr}
\DeclareMathOperator{\rank}{rank}
\DeclareMathOperator{\proj}{proj}
\DeclareMathOperator{\Hom}{Hom}
\DeclareMathOperator{\End}{End}
\DeclareMathOperator{\Ker}{Ker}
\DeclareMathOperator{\Id}{Id}
\DeclareMathOperator{\diag}{diag}
\DeclareMathOperator{\rd}{rd}
\DeclareMathOperator{\fl}{fl}
\DeclareMathOperator{\af}{af}
\DeclareMathOperator{\spann}{span}
\let\Im\relax
\DeclareMathOperator{\Im}{Im}
\let\Re\relax
\DeclareMathOperator{\Re}{Re}
\DeclareMathSymbol{\hyph}{\mathord}{AMSa}{"39}

\renewcommand{\vec}[1]{\boldsymbol{\mathbf{#1}}}
\renewcommand{\AA}{\mathcal{A}}

\DeclarePairedDelimiter\abs{\lvert}{\rvert}
\DeclarePairedDelimiter\norm{\lVert}{\rVert}
\usepackage{pgfplots}
