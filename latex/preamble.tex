\usepackage{amssymb}
\usepackage{graphicx}
\usepackage[x11names]{xcolor}
\usepackage{mathrsfs}
\usepackage[shortlabels]{enumitem}
\usepackage{mathtools}
\usepackage{microtype}
\usepackage[paper=a4paper, margin=25mm]{geometry}
\usepackage[amsmath, thmmarks]{ntheorem}
\usepackage{polski}
\usepackage{array}
\usepackage{float}
\usepackage{caption}
\usepackage{setspace}
\usepackage{tabto}
\usepackage{titlesec}
\usepackage{listings}
\usepackage{tocloft}
\usepackage{lastpage}
\usepackage{fancyhdr}
\usepackage{cancel}
\usepackage{pdfpages}
\usepackage{parskip}
\usepackage{lipsum}
\usepackage{booktabs}
\usepackage{cite}
\usepackage[final, colorlinks=true, linkcolor=red, citecolor=black, urlcolor=black]{hyperref}
\usepackage{cleveref}

\usepackage{tikz}

% Lengths
\setstretch{1.1}
\setlength\parindent{0pt}

% table of contents on the titlepage
\makeatletter
\newcommand*{\tocsamepage}{\@starttoc{toc}}
\makeatother

% http://tex.stackexchange.com/questions/22119/how-can-i-change-the-spacing-before-theorems-with-amsthm
\makeatletter
\def\thm@space@setup{%
  \thm@preskip=\parskip \thm@postskip=0pt
}
\makeatother

% Periods after section numbers
\renewcommand{\cftsecleader}{\cftdotfill{\cftdotsep}}
\renewcommand{\cftpartaftersnum}{.}
\renewcommand{\cftsecaftersnum}{.}
\renewcommand{\cftsubsecaftersnum}{.}

% Cleveref fix
\crefname{section}{}{}
\crefname{subsection}{}{}
\crefname{subsubsection}{}{}
\crefname{equation}{}{}
\crefname{theorem}{}{}
\crefname{lemma}{}{}
\crefname{example}{}{}
\crefname{definition}{}{}
\crefname{figure}{}{}
\crefname{table}{}{}
\crefname{listing}{}{}
\crefname{lstlisting}{}{}
\crefname{lstinputlisting}{}{}

% Title and titlepage
\titlelabel{\thetitle.\quad}

% Common math symbols and operators
\providecommand{\ol}{\overline}
\providecommand{\oa}{\overrightarrow}
\providecommand{\ul}{\underline}
\providecommand{\tl}{\widetilde}

\providecommand{\NN}{\mathbb{N}}
\providecommand{\ZZ}{\mathbb{Z}}
\providecommand{\QQ}{\mathbb{Q}}
\providecommand{\RR}{\mathbb{R}}
\providecommand{\CC}{\mathbb{C}}
\providecommand{\OO}{\mathcal{O}}
\providecommand{\ii}{\mathrm{i}}
\providecommand{\ee}{\mathrm{e}}
\providecommand{\dd}{\,\mathrm{d}}

\renewcommand{\vec}[1]{\boldsymbol{\mathbf{#1}}}
\newcommand{\opbox}{\mathbin{\Box}}
\newcommand{\optri}{\mathbin{\Delta}}

\DeclarePairedDelimiter\abs{\lvert}{\rvert}
\DeclarePairedDelimiter\norm{\lVert}{\rVert}

\makeatletter
\renewcommand*\env@matrix[1][*\c@MaxMatrixCols c]{%
  \hskip -\arraycolsep
  \let\@ifnextchar\new@ifnextchar
  \array{#1}}
\makeatother

% Column type for math-mode tables
\newcolumntype{C}{>{\(}c<{\)}}

% enumerating (a), (b), ...
\setlist[enumerate,1]{label={(\alph*)}}

% Footer and header
\pagestyle{fancy} 
\pagenumbering{arabic}
\fancyfoot[L]{}
\fancyfoot[C]{\thepage}
\fancyfoot[R]{}

\fancyhead[L]{}
\fancyhead[R]{}
\renewcommand{\headrulewidth}{0pt}
\renewcommand{\footrulewidth}{0pt}

% Theorem-like environments
\makeatletter
\renewtheoremstyle{break}%
    {\item[\rlap{\vbox{\hbox{\hskip\labelsep \theorem@headerfont
##1\ ##2\theorem@separator}\hbox{\strut}}}]}%
    {\item[\rlap{\vbox{\hbox{\hskip\labelsep \theorem@headerfont
##1\ ##2\theorem@separator \mdseries\ (##3)}\hbox{\strut}}}]}
\renewtheoremstyle{plain}%
  {\item[\hskip\labelsep \theorem@headerfont ##1\ ##2\theorem@separator]}%
{\item[\hskip\labelsep \theorem@headerfont ##1\ ##2\theorem@separator\  \mdseries(##3)]}
\renewtheoremstyle{nonumberplain}%
  {\item[\theorem@headerfont\hskip\labelsep ##1\theorem@separator]}%
{\item[\hskip\labelsep \theorem@headerfont ##1\ \mdseries(##3)\theorem@separator]}
\makeatother

\theoremstyle{break}
\theoremheaderfont{\bfseries}
\theorembodyfont{\mdseries}
\theoremsymbol{}
\theoremseparator{.}
\newtheorem{theorem}{Twierdzenie}[section]

\theoremstyle{break}
\theoremheaderfont{\bfseries}
\theorembodyfont{\mdseries}
\theoremsymbol{}
\theoremseparator{.}
\newtheorem{lemma}[theorem]{Lemat}

\theoremstyle{break}
\theoremheaderfont{\bfseries}
\theorembodyfont{\mdseries}
\theoremsymbol{}
\theoremseparator{.}
\newtheorem{corollary}[theorem]{Wniosek}

\theoremstyle{break}
\theoremheaderfont{\bfseries}
\theorembodyfont{\mdseries}
\theoremsymbol{}
\theoremseparator{.}
\newtheorem{definition}[theorem]{Definicja}

\theoremstyle{break}
\theoremheaderfont{\bfseries}
\theorembodyfont{\mdseries}
\theoremsymbol{}
\theoremseparator{.}
\newtheorem{example}[theorem]{Przykład}

\theoremstyle{break}
\theoremheaderfont{\bfseries}
\theorembodyfont{\mdseries}
\theoremsymbol{}
\theoremseparator{.}
\newtheorem{algorithm}[theorem]{Algorytm}

% \vspace{-\belowdisplayskip}\[\] - put this hack at the end of proof if the end-of-proof square isn't placed correctly 
\theoremstyle{nonumberplain}
\theoremheaderfont{\bfseries}
\theorembodyfont{\mdseries}
\theoremsymbol{\ensuremath\square}
\theoremseparator{:}
\newtheorem{proof}{Dowód}

\theoremstyle{plain}
\theoremheaderfont{\bfseries}
\theorembodyfont{\mdseries}
\theoremsymbol{}
\theoremseparator{.}
\newtheorem{problem}[theorem]{Zadanie}

\theoremstyle{nonumberplain}
\theoremheaderfont{\bfseries}
\theorembodyfont{\mdseries}
\theoremsymbol{}
\theoremseparator{:}
\newtheorem{solution}{Rozwiązanie}
